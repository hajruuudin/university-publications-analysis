\section{Research Questions}

The primary objective of this study is to examine the longitudinal shifts in research volume produced by elite global universities across three distinct academic domains over the past two decades. Given the rapid pace of technological innovation, the impact of a global pandemic, and various economic fluctuations during this period, it is reasonable to anticipate significant volatility in research accumulation and distribution. To explore the nature of this distribution, we have formulated three research questions, each associated with a specific investigative goal:

\begin{enumerate}
    \item \textbf{Question 1:} Has the volume of research within Medicine, Computer Science, and Business exhibited significant fluctuations over the last 20 years among the top 100 global universities?
    \begin{itemize}
        \item \textit{Goal:} To understand how technological advancements and global socio-economic shifts have influenced the trajectory of scholarly output over time.
    \end{itemize}
    
    \item \textbf{Question 2:} Is there a statistically significant correlation between a university's global ranking and its total research output?
    \begin{itemize}
        \item \textit{Goal:} To determine whether higher-ranked institutions exhibit a greater predisposition toward research-intensive activities as opposed to a primary focus on pedagogical instruction and the transmission of existing knowledge.
    \end{itemize}
    
    \item \textbf{Question 3:} How have research priorities within the field of Computer Science shifted following the emergence of Artificial Intelligence (AI) and Machine Learning?
    \begin{itemize}
        \item \textit{Goal:} To evaluate how AI—one of the most transformative developments of the 21st century—has redirected the focus of inquiry within the broader Computer Science discipline.
    \end{itemize}
\end{enumerate}

Due to the analytical nature of these questions, it is essential to define the scope of the underlying data. For Questions 1 and 2, the dataset is constrained to institutions ranked within the Top 500 as of 2026 by \textit{Times Higher Education}. For Question 3, the analysis incorporates all publicly available research papers indexed within the OpenAlex data source. Based on these parameters, we formulate the following null hypotheses ($H_0$):

\begin{quote}
    \textit{$H_{01}$: Significant differences exist in the research volume fluctuations across Medicine, Computer Science, and Business among top-ranked universities over the past two decades.}
\end{quote}

\begin{quote}
    \textit{$H_{02}$: Higher-ranking universities contribute significantly more to the volume of research papers compared to lower-ranked institutions within the contexts of Medicine, Computer Science, and Business.}
\end{quote}

\begin{quote}
    \textit{$H_{03}$: Research themes within Computer Science have undergone a paradigm shift from traditional Software Engineering to a focus on Machine Learning and Natural Language Processing (NLP), reflecting the emergence of AI in the modern era.}
\end{quote}