\section{Research Methodologies}

This study employs a quantitative, descriptive research design utilizing secondary data sources. This methodological framework was selected to ensure a robust analysis of large-scale publication trends. The following subsections define these core components and their specific applications within this research.

\subsection{Quantitative Analysis}
Quantitative research focuses on the systematic empirical investigation of observable phenomena via statistical, mathematical, or computational techniques \parencite{lakshman2000quantitative}. While traditional quantitative methods often rely on primary data collection through surveys or questionnaires to obtain high-volume responses, this study applies quantitative principles to bibliometric metadata. The objective is to analyze the numerical volume of research output rather than the qualitative content of individual papers. This data-driven approach is particularly suited for this study, as it allows for the objective measurement of research productivity across diverse disciplines and institutions \parencite{watson2015quantitative}.

\subsection{Descriptive Approach}
A descriptive research design is utilized to observe and document the current state of research distribution without manipulating the variables involved. The primary goal is to visualize trends within the selected datasets and provide a comprehensive analysis of query results. Each research question is addressed through a combination of graphical representation—generated via MATLAB—and statistical interpretation. By examining these distributions, we can effectively evaluate the validity of the three null hypotheses formulated in the preceding section.

\subsection{Secondary Data Utilization}
This study relies exclusively on secondary data, meaning the information was curated by external organizations rather than gathered directly by the author through primary experimentation. Utilizing established, open-source datasets like OpenAlex and the \textit{Times Higher Education} rankings provides access to a longitudinal and global database that would be impossible to replicate through primary collection. This secondary approach allows for extensive data manipulation and formatting, enabling a granular look at research shifts over a twenty-year horizon.