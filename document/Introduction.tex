\section{Introduction}

When the term \textit{research} is invoked, the prevailing imagery often centers on laboratory-based data analysis conducted by scientists pursuing breakthroughs in medicinal or chemical fields. However, research as a fundamental concept permeates every segment of education and specialized knowledge. Contemporary societal advancements are inextricably linked to the extensive inquiry humanity has conducted across diverse disciplines throughout history. While scholarly inquiry dates back to antiquity \cite{REFERENCE}, demonstrating its presence long before the advent of modern technology, the current pace of innovation in science, engineering, and business would be impossible without the intensive and precise research currently executed by global institutions.

Commonly, research output is associated with prestigious, high-ranking institutions that possess superior computational power and human resources. While this correlation is widely accepted, it should not marginalize or devalue the contributions and impact of lower-ranked institutions and smaller organizations. This paper investigates the magnitude of this impact and examines whether the difference in research quality and quantity between lower- and higher-ranked institutions is widening. Furthermore, it questions whether this perceived gap is a byproduct of institutional prestige rather than a true reflection of research intensity.

The primary objective of this study is to determine if higher-ranked institutions produce a significantly larger volume of research compared to their lower-ranked counterparts and to analyze how this gap has evolved over the past two decades. Additionally, this paper examines the longitudinal fluctuations in research volume within three distinct academic domains:

\begin{itemize}
    \item \textbf{Medicine:} Encompassing healthcare, chemistry, human anatomy, and psychological analysis.
    \item \textbf{Computer Science:} Including software engineering, data science, networking, and the flourishing field of Artificial Intelligence (AI).
    \item \textbf{Business:} Focusing on economics, finance, and the mechanics of economic growth and recession.
\end{itemize}

These three fields provide a comprehensive cross-section of academic inquiry. It is noted that institutions may exhibit a natural inclination toward specific disciplines; for instance, the Massachusetts Institute of Technology (MIT) typically produces a higher volume of Computer Science research compared to the University of Yonsei, Seoul, as explored in subsequent sections of this paper. Finally, we provide a detailed analysis of the growth of Computer Science research over the last 30 years, specifically focusing on the emergence of AI and Natural Language Processing (NLP).