\section{Conclusion}

Based on the empirical evidence and the statistical analysis presented in this study, we can draw definitive conclusions regarding the three research questions established at the outset of this inquiry.

The findings confirm the validity of $H_{01}$, which posited that research volume across Medicine, Computer Science, and Business in the Top 100 global universities varies significantly in magnitude. The results demonstrate that Medicine remains the predominant field, consistently yielding nearly double the research volume of Computer Science and Business combined. Furthermore, the longitudinal data suggests that this disciplinary dominance is likely to persist as healthcare remains a global priority.

Regarding $H_{02}$, the data validates the hypothesis that higher-ranking universities contribute disproportionately more to global research output compared to lower-ranking institutions. The relative share of research papers across institutional tiers has remained remarkably consistent over the last two decades. This indicates that elite universities have firmly established their roles as primary research engines, increasingly prioritizing scientific discovery over traditional pedagogical instruction.

Finally, the analysis supports $H_{03}$, illustrating that the emergence of Artificial Intelligence (AI) has catalyzed a paradigm shift within Computer Science. Topics such as Machine Learning (ML) and Natural Language Processing (NLP) have overtaken traditional fields like Software Engineering and Computer Networking in both growth rate and total publication share. This shift highlights the redirection of scholarly attention toward the burgeoning AI landscape.

\subsection{Limitations and Future Work}

Despite the robust nature of the findings, several limitations must be acknowledged. First, data for late 2024 and 2025 may be incomplete due to indexing delays at authorized institutions, which could lead to minor underestimations of recent output. Second, the synthesis of disparate data sources—namely OpenAlex and \textit{Times Higher Education}—required extensive data sanitization to mitigate sampling bias and resolve duplicate institutional identifiers. 

To build upon this work, future research could explore:
\begin{itemize}
    \item \textbf{Expanded Data Access:} Utilizing private institutional datasets or proprietary indexes (e.g., Scopus or Web of Science) to verify the volumes reported in open-source repositories.
    \item \textbf{Interdisciplinary Disambiguation:} Conducting a more granular analysis within Medicine to account for interdisciplinary overlaps, ensuring that technical developments in bioinformatics or health economics are not misattributed to a single domain.
\end{itemize}